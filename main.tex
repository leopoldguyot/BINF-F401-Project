\documentclass[a4paper, 11pt]{article}

\usepackage[utf8]{inputenc}

\usepackage[T1]{fontenc}

\usepackage[english]{babel}

\usepackage{graphicx}

\usepackage{multicol}

\usepackage{floatrow}

\usepackage[margin = 1in]{geometry}

\usepackage{float}

\usepackage[hidelinks, urlcolor=cyan]{hyperref}

\usepackage{url}

\usepackage{natbib}

\bibliographystyle{abbrvnat}
\setcitestyle{authoryear,open={(},close={)}}

\usepackage{csquotes}

\usepackage{fancyhdr}

\usepackage{lipsum}

%\addbibresource{references.bib}

\title{\Large BINF-F401 Project \\
\huge Title}


\author{
	Draguet Simon
	\and
	Godin Maximilien
	\and
	Guyot Léopold
	}

\date{\today}

\begin{document}

\pagestyle{fancy}
\setlength{\headheight}{32.3pt}
\fancyhead{}\fancyfoot{}
\fancyhead[L]{\includegraphics[scale = 0.05]{figures/LOGO_Universite _libre_bruxelles.png}}
\fancyhead[R]{Title}
\fancyfoot[R]{\thepage}

\maketitle

\begin{multicols}{2}
	bjr
\section{Introduction}
\lipsum

\section{exploration of  clinical variables}
Description des variables .....  

\subsection{Distribution of the variables }

By examining the clinical variables through histograms, we can visualize their distribution. Most of the continuous variables do not seem to follow a normal distribution. For instance, the variable AGE appears to be asymmetric, exhibiting a right skew (fig X).
To now if the continuous variable were normally distributed, we used a Shapiro-Wilk test. The results showed that AGE, HGHT, BMI, TRISCHD all had p-value under 5, leading us to reject the null hypothesis, which say that the sample is drawn form a normally distributed population. However, for WGHT, it was higher than 5 , so we didn’t reject the null hypothesis (table X ) . 

To do further analysis, we need them to be normally distributed. We tried to apply various transformations like log, square root, square. Only the square transformation successfully normalized the AGE sample. 
Because we needed it to work on all the continuous variable, we finally chose to use the rank-based inverse normal transformation (INT). that first convert the variable into ranks, then map it to a normal distribution. 

$ Y^t_{i}=  \Phi^{-1}(r-C/N-2C+1) $

where $r_i$ is the ordinary rank of the $_ith$ case among the N observations and $\Phi^{-1}$ denotes the standard normal quantile (or probit) function.For the value of C we use C=3/8(ref).

If we look at the discrete variable, we can see that we don't have a balanced distribution. For example, for the variable AGE, there are more than twice as many males as females. For DTHHRDY, we can observe that most of the deaths occurred in ventilator cases. When conducting analysis, it's essential to keep these observations in mind as they can significantly influence the interpretation. 


\begin{scriptsize}	
	
	\textbf{Associated script : \href{https://github.com/leopoldguyot/BINF-F401-Project/*}{.R}}



\end{scriptsize}

\subsection{Differential Expression Analysis}
\lipsum[4]

\section{Results and Discussion}
\lipsum
\subsection{Graphic exploration}
\lipsum[5]

\bibliography{references}

\end{multicols}
\end{document}
